% Unofficial New York University Poster Template
% a fork of https://github.com/anishathalye/gemini
% also refer to https://github.com/k4rtik/uchicago-poster



\documentclass[final]{beamer}

% ====================
% Packages
% ====================

\usepackage[T1]{fontenc}
\usepackage[utf8]{luainputenc}
\usepackage{lmodern}
\usepackage[size=custom, width=120,height=72, scale=1]{beamerposter}
\usetheme{gemini}
\usecolortheme{cam}
\usepackage{graphicx}
\usepackage{booktabs}
\usepackage{tikz}
\usetikzlibrary{positioning,arrows.meta}
\usepackage{pgfplots}
\pgfplotsset{compat=1.14}
\usepackage{anyfontsize}

% ====================
% Lengths
% ====================

% If you have N columns, choose \sepwidth and \colwidth such that
% (N+1)*\sepwidth + N*\colwidth = \paperwidth
\newlength{\sepwidth}
\newlength{\colwidth}
\setlength{\sepwidth}{0.025\paperwidth}
\setlength{\colwidth}{0.3\paperwidth}

\newcommand{\separatorcolumn}{\begin{column}{\sepwidth}\end{column}}

% ====================
% Title
% ====================

\title{[POSTER TITLE PLACEHOLDER]}

\author{[AUTHOR NAMES PLACEHOLDER]}

\institute[shortinst]{Department of [DEPARTMENT], New York University}

% ====================
% Footer (optional)
% ====================

\footercontent{
  \href{https://github.com/MengqiLiu-9543/capstone_project-33}{GitHub Repository} \hfill
  Capstone Project 2025, New York \hfill
  \href{mailto:[EMAIL]}{[EMAIL]}}
% (can be left out to remove footer)

% ====================
% Logo (optional)
% ====================

% use this to include logos on the left and/or right side of the header:
% Left: institution
\logoleft{\includegraphics[height=5cm]{logos/nyu_long_white.png}}
% Right: funding agencies and other affilations 
%\logoright{\includegraphics[height=7cm]{logos/NSF.eps}}
% ====================
% Body
% ====================

\begin{document}



\begin{frame}[t]
\begin{columns}[t]
\separatorcolumn

% ====================
% COLUMN 1
% ====================
\begin{column}{\colwidth}

  % Introduction
  \begin{block}{Introduction}
    [INTRODUCTION TEXT PLACEHOLDER - ~100 words]
    
    % Placeholder for intro figure/diagram
    \begin{figure}
      \centering
      \framebox(200,100){[Introduction Figure Placeholder]}
      \caption{[Introduction Figure Caption]}
    \end{figure}
  \end{block}

  % Task 1 Section
  \begin{block}{[TASK 1 TITLE PLACEHOLDER]}
    [TASK 1 DESCRIPTION PLACEHOLDER - ~100 words]
    
    % Placeholder for Task 1 figure/table
    \begin{figure}
      \centering
      \framebox(200,120){[Task 1 Figure/Table Placeholder]}
      \caption{[Task 1 Caption]}
    \end{figure}
  \end{block}

  % Task 2 Section
  \begin{block}{[TASK 2 TITLE PLACEHOLDER]}
    [TASK 2 DESCRIPTION PLACEHOLDER - ~100 words]
    
    % Placeholder for Task 2 figure/table
    \begin{figure}
      \centering
      \framebox(200,120){[Task 2 Figure/Table Placeholder]}
      \caption{[Task 2 Caption]}
    \end{figure}
  \end{block}

\end{column}

\separatorcolumn

% ====================
% COLUMN 2
% ====================
\begin{column}{\colwidth}

  % Task 3 Section
  \begin{alertblock}{Rare \& Unexpected Signal Detection}
    We detect rare and unexpected drug-AE relationships using Isolation Forest anomaly detection combined with a 4-step filtering pipeline: (1) statistical anomaly identification, (2) FDA label filtering, (3) indication term removal, and (4) frequency-based filtering. Results are validated using disproportionality metrics (PRR>2, IC025>0, Chi-square>4). Our pipeline identified 1,386 rare signals across 37 oncology drugs. For each signal, BERT-based clinical feature analysis identifies demographic and medical history risk factors.
    
    % Pipeline Funnel Diagram
    \begin{figure}
      \centering
      \begin{tikzpicture}[
        node distance=0.75cm,
        every node/.style={font=\scriptsize},
        box/.style={
          rectangle, rounded corners, draw=black, thick,
          align=center, minimum width=6cm, minimum height=0.75cm
        },
        arrow/.style={-{Latex[length=1.8mm]}, thick}
      ]
      \node[box] (all) {Drug--Event Pairs (FAERS)};
      \node[box, below=of all] (step1) {Isolation Forest};
      \node[box, below=of step1] (score) {Anomaly Scores};
      \node[box, below=of score] (step2) {Remove Known AEs};
      \node[box, below=of step2] (step3) {Remove Indication Terms};
      \node[box, below=of step3] (step4) {Remove High-Frequency AEs};
      \node[box, below=of step4] (final) {Rare \& Unexpected AEs};
      \draw[arrow] (all) -- (step1);
      \draw[arrow] (step1) -- (score);
      \draw[arrow] (score) -- (step2);
      \draw[arrow] (step2) -- (step3);
      \draw[arrow] (step3) -- (step4);
      \draw[arrow] (step4) -- (final);
      \end{tikzpicture}
      \caption{4-Step Filtering Pipeline: From raw data to rare signals}
    \end{figure}
    
    % Task 3 Example Outputs - Three Tables Side by Side
    \begin{figure}
      \centering
      \begin{tabular}{@{}c@{\hspace{0.2cm}}c@{\hspace{0.2cm}}c@{}}
        \raisebox{-\height}{\scalebox{0.8}{\includegraphics[width=0.25\textwidth]{table1_rare_unexpected.pdf}}} &
        \raisebox{-\height}{\scalebox{0.8}{\includegraphics[width=0.25\textwidth]{table2_not_rare.pdf}}} &
        \raisebox{-\height}{\scalebox{1.25}{\includegraphics[width=0.38\textwidth]{table3_risk_factors.pdf}}} \\[0.2cm]
        (a) Rare \& Unexpected & (b) Not Rare/Unexpected & (c) Risk Factor Analysis
      \end{tabular}
      \caption{Three Example Outputs from Task 3 Analysis}
    \end{figure}
  \end{alertblock}

  % Task 5 Section
  \begin{block}{[TASK 5 TITLE PLACEHOLDER]}
    [TASK 5 DESCRIPTION PLACEHOLDER - ~100 words]
    
    % Placeholder for Task 5 figure/table
    \begin{figure}
      \centering
      \framebox(200,120){[Task 5 Figure/Table Placeholder]}
      \caption{[Task 5 Caption]}
    \end{figure}
  \end{block}

\end{column}

\separatorcolumn

% ====================
% COLUMN 3
% ====================
\begin{column}{\colwidth}

  % Conclusion/Future Work
  \begin{exampleblock}{Conclusion \& Future Work}
    [CONCLUSION TEXT PLACEHOLDER - ~100 words]
    
    \begin{itemize}
      \item \textbf{[Future Work Item 1]}
      \item \textbf{[Future Work Item 2]}
      \item \textbf{[Future Work Item 3]}
    \end{itemize}
    
    % Placeholder for summary figure
    \begin{figure}
      \centering
      \framebox(200,100){[Summary/Overview Figure Placeholder]}
      \caption{[Summary Caption]}
    \end{figure}
  \end{exampleblock}

  % Key Results/Highlights
  \begin{block}{Key Results}
    \begin{itemize}
      \item \textbf{[Key Result 1]}
      \item \textbf{[Key Result 2]}
      \item \textbf{[Key Result 3]}
      \item \textbf{[Key Result 4]}
    \end{itemize}
  \end{block}


\end{column}

\separatorcolumn
\end{columns}
\end{frame}

\end{document}
